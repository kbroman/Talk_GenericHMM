\documentclass[aspectratio=169,12pt,t]{beamer}
\usepackage{graphicx}
\setbeameroption{hide notes}
\setbeamertemplate{note page}[plain]
\usepackage{listings}
\usepackage{eepic}

\input{header.tex}

%%%%%%%%%%%%%%%%%%%%%%%%%%%%%%%%%%%%%%%%%%%%%%%%%%%%%%%%%%%%%%%%%%%%%%
% end of header
%%%%%%%%%%%%%%%%%%%%%%%%%%%%%%%%%%%%%%%%%%%%%%%%%%%%%%%%%%%%%%%%%%%%%%

% title info
\title{general HMM for multi-parent populations}
\author{\vspace*{-10pt} \href{https://kbroman.org}{Karl Broman}}
\institute{Biostatistics \& Medical Informatics, UW{\textendash}Madison}
\date{\href{https://twitter.com/kwbroman}{\tt \scriptsize \color{foreground} @kwbroman}
\\[-4pt]
\href{https://kbroman.org}{\tt \scriptsize \color{foreground} kbroman.org}
\\[-4pt]
\href{https://github.com/kbroman}{\tt \scriptsize \color{foreground} github.com/kbroman}
\\[-4pt]
{\scriptsize \href{https://kbroman.org/Talk_GeneralHMM}{\tt kbroman.org/Talk\_GeneralHMM}}
}


\begin{document}

% title slide
{
\setbeamertemplate{footline}{} % no page number here
\frame{
  \titlepage

  \vfill \hfill
  \href{https://creativecommons.org/publicdomain/zero/1.0/}{\includegraphics[height=7mm]{Figs/cc-zero.png}}

  \note{
    These are slides for a talk for the CTC ({\tt
    www.complextrait.org/ctc2021/}) on 1 Sept 2021.

    Slides:
    \href{https://kbroman.org/Talk_GeneralHMM/general_hmm.pdf}{\tt
      kbroman.org/Talk\_GeneralHMM/general\_hmm.pdf} \\[12pt]
    Slides with notes:
    \href{https://kbroman.org/Talk_GeneralHMM/general_hmm_notes.pdf}{\tt
      kbroman.org/Talk\_GeneralHMM/general\_hmm\_notes.pdf} \\[12pt]
    Source: \href{https://github.com/kbroman/Talk_GeneralHMM}{\tt github.com/kbroman/Talk\_GeneralHMM}

  }

} }




\begin{frame}[c]{}

  \begin{columns}

    \column{0.5\textwidth}

    \centering
    {\color{title} Recombinant Inbred Lines}

    \bigskip

    \figh{Figs/ri8.pdf}{0.8}


    \column{0.5\textwidth}

    \centering
    {\color{title} Heterogeneous Stock}

    \bigskip

    \figh{Figs/hs.pdf}{0.8}

  \end{columns}


  \note{
    Multi-parent populations are experimental crosses starting from
    multiple inbred founder lines.

    Major examples include the Collaborative Cross, a set of 8-way
    recombinant inbred lines in mouse, and Hetereogeneous Stock, which
    have been developed in both mice and rats and are advanced
    intercross populations derived from 8 founders. The Diversity
    Outbred mouse population is similar to HS. In plants, multi-parent
    recombinant inbred lines are called MAGIC lines (for multiparent
    advanced generation inter-cross).

    The offspring chromosomes will be mosaics of the founder
    chromosomes. Multi-parent populations can be homozygous (like RIL)
    or heterozygous (like HS). The number of founders need not be 8.
  }

\end{frame}




\begin{frame}[c]{Genome reconstruction}


  \figh{Figs/genome_reconstr.pdf}{0.9}

  \note{

    A key step in the analysis of multi-parent populations is genome
    reconstruction: using dense SNP genotypes in the founders and MPP
    offspring to infer the haplotypes across the genome.

    Here we consider a 1 Mbp region on chromosome 14 in a single
    Diversity Outbred Mouse. Open and closed circles indicate AA and
    BB genotypes at SNPs. Gray circles indicate AB heterozygous
    genotypes. Using the SNP data along the chromosome, we can
    calculate the probability of each possible genotype at each
    position.

    For this mouse, the left half of the interval looks to be
    homozygous DD, while the right half looks to be heterozygous AD.
  }

\end{frame}

\begin{frame}[c]{}

\Large

Slides: \href{https://kbroman.org/Talk_GeneralHMM}{\tt kbroman.org/Talk\_GeneralHMM}

\vspace*{-7mm}
\hfill
\href{https://creativecommons.org/publicdomain/zero/1.0/}{\includegraphics[height=7mm]{Figs/cc-zero.png}}

\vspace{4mm}

\href{https://kbroman.org}{\tt \lolit kbroman.org}

\vspace{5mm}

\href{https://github.com/kbroman}{\tt \lolit github.com/kbroman}

\vspace{5mm}

\href{https://twitter.com/kwbroman}{\tt \lolit @kwbroman}

\vspace{5mm}

\href{https://kbroman.org/qtl2}{\tt \lolit kbroman.org/qtl2}


\note{
  Here's where you can find me and these slides.
}

\end{frame}






\end{document}
